\documentclass[12pt, letterpaper]{report}

\usepackage[utf8]{inputenc}
\usepackage[spanish]{babel}
\usepackage[left=2.54cm, top=2.54cm, bottom=2.54cm, right=2.54cm]{geometry}
\usepackage{lipsum}
\usepackage{graphicx}
\usepackage{xcolor}
\definecolor{azul}{rgb}{0.01, 0.28, 1.0}
\definecolor{codigo}{rgb}{1.0, 0.44, 0.37}
\usepackage{setspace}
\usepackage{enumerate}
\usepackage{float}

\usepackage[hidelinks]{hyperref}
\usepackage{blindtext}
\usepackage{listings} %Paquete codigo fuente

\addto\captionsspanish{
	\renewcommand{\chaptername}{Parte}
}

\usepackage{tcolorbox}

\usepackage{caption} %Quitar numero de figuras

\usepackage{listings}
\usepackage{inconsolata}

\usepackage{lmodern}

% Preambulo codigo fuente Inicio

\usepackage{minted}
\newmintedfile[myJava]{java}{
	linenos,
	numbersep=5pt,
	gobble=0,
	frame=lines,
	framesep=2mm,
	breaklines=true,
}
\newcommand{\myJavaCode}[2]{
	\myJava[label=#2.java]{#1.java}
}

\newcommand{\codigo}[1]
{
	\textcolor{codigo}{\texttt{#1}}
}


\title{Tutorial JavaFX}
\author{Mateo Mtz}

\begin{document}
	\begin{titlepage}
	\begin{center}
		\huge{\textbf{Tutorial JavaFX}}
		\\
		\vspace{3mm}
		{\LARGE \textbf{Libreta de direcciones con JavaFX}}
		\\
		\vspace{5mm}
		
		\begin{figure}[h]
			\centering
			\includegraphics{img/logo-javafx}
		\end{figure}
	
		\textcolor{azul}{\rule{\linewidth}{0.75mm}}
		\textcolor{azul}{\href{https://code.makery.ch/es/library/javafx-tutorial/}{Enlace del proyecto original}}\\
		\vspace{5mm}
		Mayo 2020
		
	\end{center}
\end{titlepage}
	\tableofcontents
	\section{Por que el proyecto}\\
El tutorial original fue escrito en inglés y traducido al español por, en este documento la intención es realizar el mismo proyecto con las actulizaciones a la última versión del JDK.
\\

\section{Introducción}\\
JavaFX proporciona a los desarrolladores de Java una nueva plataforma gráfica. JavaFX 2.0 se publicó en octubre del 2011 con la intención de reemplazar a Swing en la creación de nuevos interfaces gráficos de usuario (IGU). Cuando empecé a enseñar JavaFX en 2011 era una tecnología muy incipiente todavía. No había libros sobre JavaFX que fueran adecuados para estudiantes de programación noveles, así es que empecé a escribir una serie de tutoriales muy detallados sobre JavaFX.
	\chapter{Eclipse y Scene Builder}
\section{Contenidos en Parte 1}
\begin{itemize}
	\item Familiarizándose con JavaFX.
	\item Crear y empezar un proyecto JavaFX.
	\item Uso de Scene builder para diseñar la interfaz de usuario
	\item Estructura básica de una aplicación mediante el patrón Modelo Vista Controlador (MVC).
\end{itemize}

\section{Prerrequisitos}
\begin{itemize}
	\item Última versión de \textcolor{azul}{\href{https://www.oracle.com/java/technologies/javase-downloads.html}{Java JDK 8 o superior}}.
	\item Eclipse 4.3 o superior con el plugin e(fx)clipse. La forma más sencilla de obtenerlo es descargarse la distribución preconfigurada desde \textcolor{azul}{\href{https://efxclipse.bestsolution.at/install.html\#all-in-one}{e(fx)clipse website}}. Como alternativa puedes usar un \textcolor{azul}{\href{https://www.eclipse.org/efxclipse/install.html}{sitio de actualización}} para tu instalación de Eclipse.
	\item \textcolor{azul}{\href{https://www.oracle.com/technetwork/java/javase/downloads/javafxscenebuilder-info-2157684.html}{Scene Builder 2.0}} o superior
\end{itemize}

\section{Configuración de Eclipse}
Hay que indicarle a Eclipse que use JDK 8 y también dónde se encuentra el ejecutable del Scene Builder:
\begin{enumerate}
	\item Abre las Preferencias de Eclipse (menú\textit{ Window | Preferences} y navega hasta \textit{Java | Installed JREs}.
	\item Si no lo tienes el jre1.8 en la lista de JREs, entonces pulsa \textit{Add}…, selecciona \textit{Standard VM} y elige el Directorio de instalación (JRE Home directory) de tu JDK 8.
	\item Elimina otros JREs o JDKs de tal manera que \textbf{JDK 8 se convierta en la opción por defecto}.
	\begin{figure}[H]
		\includegraphics[height=7cm]{img/preferences-jdk}
	\end{figure}
	\item Navega a \textit{Java | Compiler}. Establece el \textbf{nivel de cumplimiento del compilador en 1.8} (Compiler compliance level).
	\begin{figure}[H]
		\includegraphics[height=7cm]{img/preferences-compliance}
	\end{figure}
	\item Navega hasta \textit{Java | JavaFX}. Especifica la ruta al ejecutable del Scene Builder.
	\begin{figure}[H]
		\includegraphics[height=7cm]{img/preferences-javafx}
	\end{figure}
	
\end{enumerate}


\section{Enlaces útiles}
Te podría interesar los siguientes enlaces:
\begin{enumerate}
	\item \textcolor{azul}{\href{https://docs.oracle.com/javase/8/docs/api/}{Java 8 API}} - Documentación (JavaDoc) de las clases estándar de Java
	\item \textcolor{azul}{\href{https://docs.oracle.com/javase/8/javafx/api/}{JavaFX 8 API}} - Documentación de las clases JavaFX
	\item  \textcolor{azul}{\href{https://controlsfx.bitbucket.io/}{ControlsFX API}} - Documentación (JavaDoc) - Documentación para el proyecto ControlsFX, el cual ofrece controles JavaFX adicionales
	\item \textcolor{azul}{\href{https://docs.oracle.com/javase/8/javafx/get-started-tutorial/get_start_apps.htm}{Oracle’s JavaFX}} Tutorials - Tutoriales oficiales de Oracle sobre JavaFX
\end{enumerate}
\texttt{¡Y ahora, manos a la obra!}

\section{Crea un nuevo proyecto JavaFX}
En Eclipse (con e(fx)clipse instalado) ve a \textit{File | New | Other…} y elige \textit{JavaFX Project}. Especifica el nombre del proyecto (ej. DireccionesApp) y aprieta \textit{Finish}.\\

	\begin{figure}[H]
		\includegraphics[width=8cm]{img/project1}
	\end{figure}
	\begin{figure}[H]
		\includegraphics[width=8cm]{img/project2}
	\end{figure}

Borra el paquete \textit{application} y su contenido que ha sido creado automáticamente.

\subsection{Crea los paquetes}
Desde el principio vamos a seguir buenos principios de diseño de software. Algunos de estos principios se traducen en el uso de la arquitectura denominada \textcolor{azul}{\href{https://es.wikipedia.org/wiki/Modelo\%E2\%80\%93vista\%E2\%80\%93controlador}{Modelo-Vista-Controlador (MVC)}}. Esta arquitectura promueve la división de nuestro código en tres apartados claramente definidos, uno por cada elemento de la arquitectura. En Java esta separación se logra mediante la creación de tres paquetes separados.\\
En el ratón hacemos clic derecho en la carpeta \textit{src, New | Package:}
\begin{itemize}
	\item \textcolor{codigo}{\texttt{ch.makery.direcciones}} - contendrá la mayoría de clases de control (C).
	\item \textcolor{codigo}{\texttt{ch.makery.direcciones.model}} - contendrá las clases del modelo (M).
	\item \textcolor{codigo}{\texttt{ch.makery.direcciones.view}} - contendrá las vistas (V)
\end{itemize}
\textbf{Nota:} Nuestro paquete dedicado a las vistas contendrá también algunos controladores dedicados exclusivamente a una vista. Les llamaremos \textbf{controladores-vista}.

\section{Crea el archivo FXML de diseño}
Hay dos formas de crear la interfaz de usuario. Programándolo en Java o mediante un archivo XML. Si buscas en Internet encontrarás información relativa a ambos métodos. Aquí usaremos XML (archivo con la extensión .fxml) para casi todo. Encuentro más claro mantener el controlador y la vista separados entre sí. Además, podemos usar la herramienta de edición visual Scene Builder, la cual nos evita tener que trabajar directamente con el XML.\\
Haz clic derecho el paquete \textit{view} y crea un nuevo archivo FXML (\textit{New | Other | FXML | New FXML Document}) llamado \textcolor{codigo}{\texttt{PersonaOverview}}.
\begin{figure}[H]
	\includegraphics[width=8cm]{img/fxmlDocument}
\end{figure}
\begin{figure}[H]
	\includegraphics[width=8cm]{img/fxmlNombreDocument}
\end{figure}

\section{Diseño mediante Scene Builder}
\newtcolorbox{mybox}[1]{colback=red!5!white,
	colframe=red!75!black,fonttitle=\bfseries,
	title=#1}
\tcbset{colback=blue!5!white,colframe=blue!75!black}
\begin{tcolorbox}[leftrule=3mm]
	\textbf{Nota:} Si no puedes hacerlo funcionar, descarga las fuentes para esta parte del tutorial e inténtalo con el archivo fxml incluido.
\end{tcolorbox}

Haz clic derecho sobre PersonaOverview.fxml y elige \textit{Open with Scene Builder}. Ahora deberías ver el Scene Builder con un AnchorPane (visible en la jerarquía de componentes (herramienta Hierarchy) situada a la izquierda).
\begin{figure}[H]
	\includegraphics[width=7cm]{img/openSceneBuilder}
\end{figure}
Ya abierto Scene Builder
\begin{enumerate}
	\item Selecciona el \textit{AnchorPane} en tu jerarquía y ajusta el tamaño en el apartado Layout (a la derecha):
	\begin{figure}[H]
		\includegraphics[width=11cm]{img/anchorPane}
	\end{figure}
	\item Añade un \textit{SplitPane (Horizontal Flow)} arrastrándolo desde la librería (Library) al área principal de edición. Haz clic derecho sobre el SplitPane en la jerarquía y elige Fit to Parent.
	\begin{figure}[H]
		\includegraphics[width=11cm]{img/splitPane}
	\end{figure}

	\item Arrastra un TableView (bajo Controls) al lado izquierdo del SplitPane. Selecciona la TableView (no una sola columna) y establece las siguientes restricciones de apariencia (Layout) para la TableView. Dentro de un AnchorPane siempre se pueden establecer anclajes (anchors) para los cuatro bordes (\textcolor{azul}{\href{https://docs.oracle.com/javase/8/javafx/layout-tutorial/builtin_layouts.htm}{más información sobre Layouts}}).
		\begin{figure}[H]
		\includegraphics[width=14cm]{img/TableView}
	\end{figure}
	
	\item Ve al menú \textit{Preview | Show Preview in Window} para comprobar si se visualiza correctamente. Intenta cambiar el tamaño de la ventana. La TableView debería ajustar su tamaño al tamaño de la ventana, pues está “anclada” a sus bordes.
	
	\item Cambia el texto de las columnas (en \textit{Properties}) a “Nombre” y “Apellido”.
	\begin{figure}[H]
		\includegraphics[width=10cm]{img/nombreApellido}
	\end{figure}
	\item Selecciona la \textit{TableView} y elige \textit{constrained-resize} para la \textit{Column Resize Policy} (en \textit{Properties}). Esto asegura que las columnas usarán siempre todo el espacio que tengan disponible.
	\begin{figure}[H]
		\includegraphics[width=7cm]{img/constraintRezise}
	\end{figure}
	\item Añade una \textit{Label} al lado derecho del \textit{SplitPane} con el texto “Detalles de Persona” (truco: usa la búsqueda en la librería para encontrar el componente \textit{Label}). Ajusta su apariencia usando anclajes.
	\begin{figure}[H]
		\includegraphics[width=13cm]{img/detallesPersona}
	\end{figure}
	\item Añade un \textit{GridPane} al lado derecho, selecciónalo y ajusta su apariencia usando anclajes (superior, derecho e izquierdo).
	\begin{figure}[H]
		\includegraphics[width=16cm]{img/gridPane}
	\end{figure}
	\item Añade las siguientes etiquetas (\textit{Label}) a las celdas del GridPane.\\
	Nota: Para añadir una fila al \textit{GridPane} selecciona un número de fila existente (se volverá de color amarillo), haz clic derecho sobre el número de fila y elige “Add Row”.
	\begin{figure}[H]
		\includegraphics[width=14cm]{img/gridPaneAddRow}
	\end{figure}
	\item Añade tres botones a la parte inferior. Truco: Selecciónalos todos, haz click derecho e invoca \textit{Wrap In | HBox}. Esto los pondrá a los 3 juntos en un \textit{HBox}. Podrías necesitar establecer un espaciado spacing dentro del HBox. Después, establece también anclajes (derecho e inferior) para que se mantengan en el lugar correcto.
	\begin{figure}[H]
		\includegraphics[width=10cm]{img/hBox}
	\end{figure}
	\begin{figure}[H]
		\includegraphics[width=10cm]{img/hBoxLayout}
	\end{figure}
	\item Ahora deberías ver algo parecido a lo siguiente. Usa el menú \textit{Preview} para comprobar su comportamiento al cambiar el tamaño de la ventana.
	\begin{figure}[H]
		\includegraphics[width=14cm]{img/previewUno}
	\end{figure}


	\begin{tcolorbox}[leftrule=3mm]
		\textbf{Nota:} Guarda los cambios en SceneBuilder, el archivo se actualizará automáticamente en Eclipse, si esto no sucede, puedes presionar F5 para actualizar el archivo.
	\end{tcolorbox}

\end{enumerate}

\section{Crea la aplicación principal}
Necesitamos otro archivo FXML para nuestra vista raíz, la cual contendrá una barra de menús y encapsulará la vista recién creada \textcolor{codigo}{\texttt{PersonaOverview.fxml}}.
\begin{enumerate}
	\item Crea otro archivo FXML dentro del paquete view llamado \textcolor{codigo}{\texttt{RootLayout.fxml}}. Esta vez, elige BorderPane como elemento raíz.
	\begin{figure}[H]
		\includegraphics[width=11cm]{img/rootLayout}
	\end{figure}
	\item Abre \textcolor{codigo}{\texttt{RootLayout.fxml}} en el Scene Builder.
	\item Cambia el tamaño del \textit{BorderPane} con la propiedad \textit{\textit{Pref Width}} establecida en 600 y \textit{Pref Height} en 400.
	\begin{figure}[H]
		\includegraphics[width=9cm]{img/rootLayoutSize}
	\end{figure}
	\item Añade una \textit{MenuBar} en la ranura superior del \textit{\textit{BorderPane}}. De momento no vamos a implementar la funcionalidad del menú.
	\begin{figure}[H]
		\includegraphics[width=9cm]{img/menuBarTop}
	\end{figure}
	
\end{enumerate}

	\chapter{Modelo y TableView}
\begin{figure}[H]
	\includegraphics[width=12cm]{img/terceraParte}
\end{figure}
\section{Contenidos en Parte 2}
\begin{itemize}
	\item Creación de una clase para el modelo.
	\item Uso del modelo en una ObservableList.
	\item Visualización del modelo mediante TableView y Controladores.
\end{itemize}

\section{Crea la clase para el Modelo}
Neceistamos un modelo para contener la información sobre los contactos de nuestra agenda. Añade una nueva clase al paquete encargado de contener los modelos (\textcolor{codigo}{\texttt{ch.makery.direcciones.model}} ) denominada \textcolor{codigo}{\texttt{Persona}}. La clase \textcolor{codigo}{\texttt{Persona}} tendrá atributos (instancias de clase) para el nombre, la dirección y la fecha de nacimiento. Añade el código siguiente a la clase. Explicaré detalles de JavaFX después del código.\\
\textcolor{codigo}{\texttt{Persona.java}}
\begin{minted}[
linenos,
numbersep=5pt,
gobble=0,
frame=lines,
framesep=2mm,
breaklines=true,
]{java}

package ch.makery.direcciones.model;

import java.time.LocalDate;

import javafx.beans.property.IntegerProperty;
import javafx.beans.property.ObjectProperty;
import javafx.beans.property.SimpleIntegerProperty;
import javafx.beans.property.SimpleObjectProperty;
import javafx.beans.property.SimpleStringProperty;
import javafx.beans.property.StringProperty;

/**
* Modelo de la clase persona
* @author Mateo Mtz
*
*/
public class Persona {
	private StringProperty nombre;
	private StringProperty apellido;
	private StringProperty calle;
	private IntegerProperty codigoPostal;
	private StringProperty ciudad;
	private ObjectProperty<LocalDate> onomastico;
	public Persona(String nombre, String apellido){
		this.nombre = new SimpleStringProperty(nombre);
		this.apellido = new SimpleStringProperty(apellido);
		
		//
		this.calle = new SimpleStringProperty("Alguna calle");
		this.codigoPostal = new SimpleIntegerProperty(1234);
		this.ciudad = new SimpleStringProperty("Alguna ciudad");
		this.onomastico = new SimpleObjectProperty<LocalDate>(LocalDate.of(1999,  2,  21));
	}
	public StringProperty getNombre() {
		return nombre;
	}
	public void setNombre(String nombre) {
		this.nombre = new SimpleStringProperty(nombre);
	}
	public StringProperty getApellido() {
		return apellido;
	}
	public void setApellido(String apellido) {
		this.apellido = new SimpleStringProperty(apellido);
	}
	public StringProperty getCalle() {
		return calle;
	}
	public void setCalle(String calle) {
		this.calle = new SimpleStringProperty(calle);
	}
	public IntegerProperty getCodigoPostal() {
		return codigoPostal;
	}
	public void setCodigoPostal(int codigoPostal) {
		this.codigoPostal = new SimpleIntegerProperty(codigoPostal);
	}
	public StringProperty getCiudad() {
		return ciudad;
	}
	public void setCiudad(String ciudad) {
		this.ciudad = new SimpleStringProperty(ciudad);
	}
	public ObjectProperty<LocalDate> getOnomastico() {
		return onomastico;
	}
	public void setOnomastico(LocalDate onomastico) {
		this.onomastico = new SimpleObjectProperty<LocalDate>(onomastico);
	}
}

\end{minted}

\subsection{Explicación del código}
\begin{itemize}
	\item Con JavaFX es habitual usar \textcolor{codigo}{\texttt{Propiedades}} para todos los atributos de un clase usada como modelo. Una \textcolor{codigo}{\texttt{Propiedad}} permite, entre otras cosas, recibir notificaciones automáticamente cuando el valor de una variable cambia (por ejemplo, si cambia \textcolor{codigo}{\texttt{apellido}}. Esto ayuda a mantener sincronizados la vista y los datos. Para aprender más sobre \textcolor{codigo}{\texttt{Propiedades}} lee \textcolor{azul}{\href{https://docs.oracle.com/javase/8/javafx/properties-binding-tutorial/binding.htm}{Using JavaFX Properties and Binding}}.
	\item \textcolor{codigo}{\texttt{LocalDate}}, el tipo que usamos para especificar la fecha de nacimiento (\textcolor{codigo}{\texttt{onomastico}}) es parte de la nueva \textcolor{azul}{\href{https://docs.oracle.com/javase/tutorial/datetime/iso/}{API de JDK 8 para la fecha y la hora}}.
\end{itemize}

\section{Una lista de personas}
Los principales datos que maneja nuestra aplicación son una colección de personas. Vamos a crear una lista de objetos de tipo \textcolor{codigo}{\texttt{Persona}} dentro de la clase principal \textcolor{codigo}{\texttt{MainApp}}. El resto de controladores obtendrá luego acceso a esa lista central dentro de \textcolor{codigo}{\texttt{MainApp}}.

\subsection{Lista observable (ObservableList)}
Estas clases gráficas de JavaFX que necesitan ser informadas sobre los cambios en la lista de personas. Esto es importante, pues de otro modo la vista no estará sincronizada con los datos. Para estos fines, JavaFX ha introducido \textcolor{azul}{\href{https://docs.oracle.com/javase/8/javafx/collections-tutorial/collections.htm}{nuevas clases de colección}}.\\
De esas colecciones, necesitamos la denominada \textcolor{codigo}{\texttt{ObservableList}}. Para crear una nueva \textcolor{codigo}{\texttt{ObservableList}}, añade el código siguiente al principio de la clase \textcolor{codigo}{\texttt{MainApp}}. También añadiremos un constructor para crear datos de ejemplo y un método de consulta (get) público:\\
\textcolor{codigo}{\texttt{MainApp.java}}
\begin{minted}[
linenos,
numbersep=5pt,
gobble=0,
frame=lines,
framesep=2mm,
breaklines=true,
]{java}

	/**
	* Lista de personas en una ObservableList
	*/
	private ObservableList<Persona> personData = FXCollections.observableArrayList();
	/**
	* Constructor
	*/
	public MainApp(){
		//Agregar algunas personas a la lista
		personData.add(new Persona("Juan", "Perez"));
		personData.add(new Persona("Pedro", "Perez"));
		personData.add(new Persona("Pablo", "Perez"));
	}
	/**
	* Devuelve los datos como una lista observable de personas.
	* @return
	*/
	public ObservableList<Persona> getPersonData(){
		return personData;
	}
\end{minted}
\section{PersonOverviewController}
Finalmente vamos a añadir datos a nuestra table. Para ello necesitaremos un controlador específico para la vista \textcolor{codigo}{\texttt{PersonOverview.fxml}}.
\begin{enumerate}
	\item Crea una clase normal dentro del paquete \textbf{view} denominado \textcolor{codigo}{\texttt{PersonOverviewController.java}}. (Debemos ponerlo en el mismo paquete que \textcolor{codigo}{\texttt{PersonOverview.fxml}} o el Scene Builder no lo encontrará, al menos no en la versión actual).
	\item Añadiremos algunos atributos para acceder a la tabla y las etiquetas de la vista. Estos atributos irán precedidos por la anotación \textcolor{codigo}{\texttt{@FXML}}. Esto es necesario para que la vista tenga acceso a los atributos y métodos del controlador, incluso aunque sean privados. Una vez definida la vista en fxml, la aplicación se encargará de rellenar automáticamente estos atributos al cargar el fxml. Así pues, añade el código siguiente:
\end{enumerate}
\begin{tcolorbox}[leftrule=3mm]
	\textbf{Nota:} acuérdate siempre de importar \textbf{javafx}, NO AWT ó Swing!.
\end{tcolorbox}

\begin{minted}[
linenos,
numbersep=5pt,
gobble=0,
frame=lines,
framesep=2mm,
breaklines=true,
]{java}
package ch.makery.direcciones.view;

import ch.makery.direcciones.MainApp;
import ch.makery.direcciones.model.Persona;
import javafx.fxml.FXML;
import javafx.scene.control.Label;
import javafx.scene.control.TableColumn;
import javafx.scene.control.TableView;

public class PersonOverviewController {
	@FXML
	private TableView<Persona> personTable;
	@FXML
	private TableColumn<Persona, String> nombresColumna;
	@FXML
	private TableColumn<Persona, String> apellidosColumna;
	
	@FXML
	private Label nombreLabel;
	@FXML
	private Label apelidoLabel;
	@FXML
	private Label calleLabel;
	@FXML
	private Label codigoPostaLabel;
	@FXML
	private Label ciudadLabel;
	@FXML
	private Label onomasticoLabel;
	
	//Referencia a la clase MainApp
	private MainApp mainApp;
	/**
	* Constructor
	* Se llama al constructor antes del método initialize()
	*/
	public PersonOverviewController(){
	}
	/**
	* Inicializa la clase de controlador
	* Este método se llama automáticamente después de cargar el archivo fxml.
	*/
	@FXML
	private void initialize(){
		//Inicialice la tabla de personas con las dos columnas.
		nombresColumna.setCellValueFactory(cellData -> cellData.getValue().getNombre());
		apellidosColumna.setCellValueFactory(cellData -> cellData.getValue().getApellido());
	}
	/**
	* Es llamado por la aplicación principal para devolverse una referencia a sí mismo.
	* @param mainApp
	*/
	public void setMainApp(MainApp mainApp){
		this.mainApp = mainApp;
		personTable.setItems(mainApp.getPersonData());
	}
}

\end{minted}
Este código necesita cierta explicación:
\begin{itemize}
	\item Los campos y métodos donde el archivo fxml necesita acceso deben ser anotados con \textcolor{codigo}{\texttt{@FXML}}. En realidad, sólo si son privados, pero es mejor tenerlos privados y marcarlos con la anotación.
	\item El método \textcolor{codigo}{\texttt{initialize()}} es invocado automáticamente tras cargar el fxml. En ese momento, todos los atributos FXML deberían ya haber sido inicializados.
	\item El método \codigo{setCellValueFactory(...)} que aplicamos sobre las columnas de la tabla se usa para determinar qué atributos de la clase \codigo{Persona} deben ser usados para cada columna particular. La flecha \codigo{->} indica que estamos usando una característica de Java 8 denominada \textit{Lambdas}. Otra opción sería utilizar un \href{https://docs.oracle.com/javase/8/javafx/api/}{\codigo{\underline{PropertyValueFactory}}}, pero entonces no ofrecería seguridad de tipo (\textit{type-safe}).
\end{itemize}

\subsection{Conexión de MainApp con PersonOverviewController}
El método \codigo{setMainApp(...)} debe ser invocado desde la clase \codigo{MainApp}. Esto nos da la oportunidad de acceder al objeto \codigo{MainApp} para obtener la lista de \codigo{Persona} y otras cosas. Sustituye el método \codigo{showPersonOverview()} con el código siguiente, el cual contiene dos líneas adicionales:\\
\codigo{MainApp.java} - nuevo método \codigo{showPersonOverview()}
\begin{minted}[
linenos,
numbersep=5pt,
gobble=0,
frame=lines,
framesep=2mm,
breaklines=true,
]{java}
	/**
	* Muestra la descripción general de la persona dentro del diseño raíz.
	*/
	public void showPersonaOverview(){
		try{
			//Carga datos de persona
			FXMLLoader loader = new FXMLLoader();
			loader.setLocation(MainApp.class.getResource("view/PersonaOverview.fxml"));
			AnchorPane personaOverview = (AnchorPane) loader.load();
			
			//Carga los datos de la persona en el centro del diseño raíz.
			rootLayout.setCenter(personaOverview);
			
			//Darle al controlador acceso a la App - lineas adicionales
			PersonOverviewController controller = loader.getController();
			controller.setMainApp(this);
		}catch (IOException e) {
			e.printStackTrace();
		}
	}
\end{minted}

\section{Vincular la vista al controlador}
¡Ya casi lo tenemos! Pero falta un detalle: no le hemos indicado a la vista declarada en \codigo{PersonOverview.fxml} cuál es su controlador y que elemento hacer corresponder to cada uno de los atributos en el controlador.

\begin{enumerate}
	\item \codigo{PersonOverview.fxml} en \textit{SceneBuilder}.
	\item Abre la sección \textit{Controller} en el lado izquierdo y selecciona \codigo{PersonOverviewController} como \textbf{controlador}.
	\begin{figure}[H]
		\includegraphics[width=8cm]{img/selectController}
	\end{figure}
	\item Selecciona \codigo{TableView} en la sección \textit{Hierarchy} y en el apartado \textit{Code} escribe o selecciona \codigo{personTable} en la propiedad \textbf{fx:id}.
	\begin{figure}[H]
		\includegraphics[width=8cm]{img/personTable}
	\end{figure}
	\item Haz lo mismo para las columnas, poniendo \codigo{nombre} y \codigo{apellido} con sus \textbf{fx:id} respectivamente.
	\item Para \textbf{cada etiqueta} en la segunda columna, introduce el \textbf{fx:id} que corresponda.
	\begin{figure}[H]
		\includegraphics[width=12cm]{img/nombreLabel}
	\end{figure}
	\item Importante: En Eclipse \textbf{refresca el projecto DireccionesApp} (tecla F5). Esto es necesario porque a veces Eclipse no se da cuenta de los cambios realizados desde el \textit{Scene Builder}.
\end{enumerate}

\section{Inicia la aplicación}
Ahora, cuando ejecutes la aplicación, deberías obtener algo parecido a la captura de pantalla incluida al principio de este artículo.\\

Enhorabuena!




\end{document}