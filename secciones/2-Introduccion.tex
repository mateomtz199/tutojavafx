\section{Por que el proyecto}
El tutorial original fue escrito en inglés y traducido al español por \textcolor{azul}{\href{https://about.me/magomar}{Mario Gómez Martínez}}, en este documento la intención es realizar el mismo proyecto con las actulizaciones a la última versión del JDK, la codificación se actualizará al español con los respectivos comentarios. 
\section{Introducción}
JavaFX proporciona a los desarrolladores de Java una nueva plataforma gráfica. JavaFX 2.0 se publicó en octubre del 2011 con la intención de reemplazar a Swing en la creación de nuevos interfaces gráficos de usuario (IGU). Cuando empecé a enseñar JavaFX en 2011 era una tecnología muy incipiente todavía. No había libros sobre JavaFX que fueran adecuados para estudiantes de programación noveles, así es que empecé a escribir una serie de tutoriales muy detallados sobre JavaFX.\\
El tutorial te guía a lo largo del diseño, programación y publicación de una aplicación de contactos (libreta de direcciones) mediante JavaFX. Este es el aspecto que tendrá la aplicación final:\\
\textbf{Colocar la imagen después}\\
\section{Lo que aprenderás}
\begin{itemize}
	\item Creación de un nuevo projecto JavaFX
	\item Uso de Scene Builder para diseñar la interfaz de usuario
	\item Estructuración de una aplicación según el patrón MVC (Modelo, Vista, Controlador)
	\item Uso de ObservableList para la actualización automática de la interfaz de usuario
	\item Uso de TableView y respuesta a cambios de selección en la tabla
	\item Creación de un diálogo personalizado para editar personas
	\item Validación de la entrada del usuario
	\item Aplicación de estilos usando CSS
	\item Persistencia de datos mediante XML
	\item Guardado del último archivo abierto en las preferencias de usuario
	\item Creación de un gráfico JavaFX para mostrar estadísticas
	\item Despliegue de una aplicación JavaFX nativa
\end{itemize}

Después de completar esta serie de tutoriales deberías estar preparado para desarrollar aplicaciones sofisticadas con JavaFX.
\section{Cómo usar este tutorial}
Hay dos formas de utilizarlo
\begin{itemize}
	\item \textbf{Máximo-aprendizaje:} Crea tu propio proyecto JavaFX desde cero.
	\item \textbf{Máxima-rápidez:} Importa el código fuente de una parte del tutorial en tu entorno de desarrollo favorito (es un proyecto Eclipse, pero puedes usar otros entornos, como Netbeans, con ligeras modificaciones). Después revisa el tutorial para entender el código. Este enfoque también resulta útil si te quedas atascado en la creación de tu propio código.
\end{itemize}
