\chapter{Modelo y TableView}

\section{Contenidos en parte 3}
\begin{itemize}
    \item Respuesta a cambios en la selección dentro de la tabla.
    \item Añade funcionalidad de los botones añadir, editar, y borrar.
    \item Crear un diálogo emergente (popup dialog) a medida para editar un contacto.
    \item Validación de la entrada del usuario.
\end{itemize}

\section{Respuesta a cambios en la selección de la Tabla}
Todavía no hemos usado la parte derecha de la interfaz de nuestra aplicación. 
La intención es usar esta parte para mostrar los detalles de la persona seleccionada por 
el usuario en la tabla. \\
En primer lugar vamos a añadir un nuevo método dentro de \codigo{PersonOverviewController} que nos ayude a 
rellenar las etiquetas con los datos de una sola persona. \\

Crea un método llamado \codigo{showPersonaDetails(Persona persona)}. Este método recorrerá todas las etiquetas y 
establecerá el texto con detalles de la persona usando \codigo{setText(...)}. Si en vez de una instancia de 
\codigo{Person} se pasa null entonces las etiquetas deben ser borradas. \\
\codigo{PersonOverviewController.java} nuevo método \codigo{showPersonaDetails(Persona persona)}

\begin{minted}[
    linenos,
    numbersep=5pt,
    gobble=0,
    frame=lines,
    framesep=2mm,
    breaklines=true,
    ]{java}

    /**
	 * Rellena todos los campos de texto para mostrar detalles sobre la persona.
	 * Si la persona especificada es nula, se borran todos los campos de texto.
	 * @param persona
	 */
	private void showPersonaDetails(Persona persona){
		if(persona != null){
			//Rellene las etiquetas con información del objeto persona.
			nombreLabel.setText(persona.getNombre().get());
			apelidoLabel.setText(persona.getApellido().get());
			calleLabel.setText(persona.getCalle().get());
			codigoPostaLabel.setText(persona.getCodigoPostal().get() + "");
			ciudadLabel.setText(persona.getCiudad().get());
			//¡Necesitamos una forma de convertir el cumpleaños en una Cadena!
			//onomasticoLabel.setText(...);
			onomasticoLabel.setText(DateUtil.format(persona.getOnomastico().get()));
		}else {
			//Si la persona es nula, quitar todo el texto
			nombreLabel.setText("");
			apelidoLabel.setText("");
			calleLabel.setText("");
			codigoPostaLabel.setText("");
			ciudadLabel.setText("");
			onomasticoLabel.setText("");
		}
	}

\end{minted}